%%%%%%%%%%%%%%%%%%%%%%%%%%%%%%%%%%%%%%%%%
% The Legrand Orange Book
% LaTeX Template
% Version 2.0 (9/2/15)
%
% This template has been downloaded from:
% http://www.LaTeXTemplates.com
%
% Mathias Legrand (legrand.mathias@gmail.com) with modifications by:
% Vel (vel@latextemplates.com)
%
% License:
% CC BY-NC-SA 3.0 (http://creativecommons.org/licenses/by-nc-sa/3.0/)
%
% Compiling this template:
% This template uses biber for its bibliography and makeindex for its index.
% When you first open the template, compile it from the command line with the 
% commands below to make sure your LaTeX distribution is configured correctly:
%
% 1) pdflatex main
% 2) makeindex main.idx -s StyleInd.ist
% 3) biber main
% 4) pdflatex main x 2
%
% After this, when you wish to update the bibliography/index use the appropriate
% command above and make sure to compile with pdflatex several times 
% afterwards to propagate your changes to the document.
%
% This template also uses a number of packages which may need to be
% updated to the newest versions for the template to compile. It is strongly
% recommended you update your LaTeX distribution if you have any
% compilation errors.
%
% Important note:
% Chapter heading images should have a 2:1 width:height ratio,
% e.g. 920px width and 460px height.
%
%%%%%%%%%%%%%%%%%%%%%%%%%%%%%%%%%%%%%%%%%

%----------------------------------------------------------------------------------------
%	PACKAGES AND OTHER DOCUMENT CONFIGURATIONS
%----------------------------------------------------------------------------------------

\documentclass[11pt,fleqn]{book} % Default font size and left-justified equations

%----------------------------------------------------------------------------------------

\input{structure} % Insert the commands.tex file which contains the majority of the structure behind the template

\begin{document}

%----------------------------------------------------------------------------------------
%	TITLE PAGE
%----------------------------------------------------------------------------------------

\begingroup
\thispagestyle{empty}
\begin{tikzpicture}[remember picture,overlay]
\coordinate [below=12cm] (midpoint) at (current page.north);
\node at (current page.north west)
{\begin{tikzpicture}[remember picture,overlay]
\node[anchor=north west,inner sep=0pt] at (0,0) {\includegraphics[width=\paperwidth]{cover}}; % Background image
\draw[anchor=north] (midpoint) node [below=6cm,fill=white!30!white,fill opacity=0.8,text opacity=1,inner sep=1cm]{\Huge\centering\bfseries\sffamily\parbox[c][][t]{\paperwidth}{\centering Developing Proven Software with B\\[25pt] % Book title
{\huge DESKBOOK}\\[40pt] % Subtitle
{\Large Thierry Lecomte, Marcel Oliveira editors}}}; % Author name
\end{tikzpicture}};
\end{tikzpicture}
\vfill
\endgroup

%----------------------------------------------------------------------------------------
%	COPYRIGHT PAGE
%----------------------------------------------------------------------------------------

\newpage
~\vfill
\thispagestyle{empty}

\noindent Copyright \copyright\ 2019 Thierry Lecomte and Marcel Oliveira\\ % Copyright notice

\noindent \textsc{Published by Thierry Lecomte and Marcel Oliveira}\\ % Publisher

\noindent \textsc{www.methode-b.com}\\ % URL

\noindent Licensed under the Creative Commons Attribution-NonCommercial 3.0 Unported License (the ``License''). You may not use this file except in compliance with the License. You may obtain a copy of the License at \url{http://creativecommons.org/licenses/by-nc/3.0}. Unless required by applicable law or agreed to in writing, software distributed under the License is distributed on an \textsc{``as is'' basis, without warranties or conditions of any kind}, either express or implied. See the License for the specific language governing permissions and limitations under the License.\\ % License information

\noindent \textit{Edition March 2019 } % Printing/edition date

%----------------------------------------------------------------------------------------
%	TABLE OF CONTENTS
%----------------------------------------------------------------------------------------

\chapterimage{back2.jpg} % Table of contents heading image

\pagestyle{empty} % No headers

\tableofcontents % Print the table of contents itself

\cleardoublepage % Forces the first chapter to start on an odd page so it's on the right

\pagestyle{fancy} % Print headers again

%---------------------------------------------------------------------
%	FOREWORD
%---------------------------------------------------------------------
\chapterimage{back2.jpg} % Chapter heading image
\chapter{Foreword}
\chapterimage{back2.jpg} % Chapter heading image
\chapter{Preface}
\label{preface}

This book provides an introduction to the CLEARSY Safety Platform (CSSP in short).
It is aimed at easing the development and the deployment of safety critical applications, up to SIL4. It is made of an integrated software development environment (IDE) and a hardware platform that natively integrates safety principles.
It relies on the smart integration of formal methods (including mathematical proof), redundant code generation and compilation, and a hardware platform that ensures a safe execution of the software.

The B formal method is at the core of the software development process. Mathematical proof ensures that the software complies with its specification (functional model) and guarantees the absence of programming errors while avoiding unit testing and integration testing. Moreover only one functional model is used to produce automatically the redundant software, avoiding the need to have two independent teams for its development \footnote{as required by railways standards for the highest SILs, for examples}.

The safety principles are built-in, both at software level and at hardware level (2oo2 hardware, 4oo4 software\footnote{put for "2 out of 2" and "4 out of 4", are consensus voting principles used for safety architectures, where all processors have to obtain the results to initiate a potentially dangerous action.}). They are out of reach of the developer who cannot alter them. The detection of any divergent behaviour among the two processors (PIC32 micro-controllers) and the four instances of the software is handled by the platform. The safety verification includes cross checks between software instances and between micro-controllers, memory integrity, ability to control outputs, etc. 

\begin{remark}
The CSSP Starter kits SK0 and SK1 are only for education and industrial prototyping respectively. If the software generated by the Atelier CSSP is the one that will be running on the final safety critical system, the electronics of SK0 and SK1 do not comply with SIL3 / SIL4 requirements. One of the reasons is that such electronics would largely increase the board prices and it would be clearly against the dissemination of the technology. \\
However the CSSP Core Module (available by the end of 2019), a daughter board to be plugged on a motherboard with digital IOs, will be SIL4-ready and usable in a real safety critical application.
\end{remark}

\section{Tool support}\index{Tool support}
Working with the CSSP \footnote{https://www.clearsy.com/en/our-tools/clearsy-safety-platform/} requires, as a minimum, access to the Atelier CSSP, an IDE derived from Atelier B \footnote{https://www.atelierb.eu/en/} and extended with specific features like diverse code generation and CSSP starter kit configuration. This IDE allows to create a CSSP project, specify and implement the behaviour of the CSSP, typecheck and compile the project. Finally the IDE allows to upload the program on the CSSP and to monitor its execution. For the execution of the program, either CSSP starter kit 0 (SK$_0$) or 1 (SK$_1$) board may be used. The only difference between the two boards are the number of digital IOs (5 for SK$_0$, 28 for SK$_1$). 


\section{Who this book is for}\index{Who this book is for}
This book is intended primarily as a textbook for post-graduates courses . It is also appropriate for courses on formal methods in general and on (safety related) embedded systems. The book assumes no prior knowledge of formal methods or of reasoning about programs. However it assumes a previous exposure to logic and a basic ability to manipulate simple logic and set theoretic expressions. No prior knowledge of the modelling language, namely the B language, is required as language elements will be introduced when needed. Moreover, as project skeletons are automatically generated, being able to develop a full B project is not required: programming the CSSP requires one component (the user\_logic operation) to be modified and possibly new components to be added. 

\section{To the instructor}\index{To the instructor}
This book has grown out of a course based around the B method and the CLEARSY Safety Platform, developed over a period of three years in Brazil, Canada, France, Italy, Portugal and UK.
The material is organised to introduce the central ideas as quickly as possible. This allows the students to become familiar with the tool support at the earliest possible stage, and to use it to develop their own programs. It also means that the students learn the B-method from a software engineering point of view, as they are taught from the viewpoint of using the B-method, rather than of the theory that underlies it. Such theory and language elements are introduced as and when they are needed. The Atelier CSSP generates a pre-filled B project, so the students do not need to develop a full B project but only to complete the specification and implementation of some CONSTANTS, VARIABLES, and OPERATIONS.
The first part of the book introduces the overall picture, the development process, and the language elements. They have to be read in this order.
The second part of the book contains a number of examples. The first two are related to synchronous and combinatorial programming, they have to be completed before moving to the next ones. \\

\noindent For any further information, requests or questions, please contact:
\begin{center}
contact-csp@clearsy.com    
\end{center}


\section{Book organisation}\index{Book organisation}

This book has an associated website, accessible from
\begin{center}
https://www.clearsy.com/en/our-tools/clearsy-safety-platform/    
\end{center}

\noindent This website contains the source code of all the examples and exercises presented in the book.

\section{Acknowledgements}\index{Acknowledgements}
The CLEARSY Safety Platform is being developed in order to compete on the international scene of safety critical systems, with the key idea to lower development and certification costs. Its development is a collective effort being produced in-house during development but also on site all over the world during deployment and exploitation, to obtain finally a safety, SIL4-ready product.\\
\\
We would like to thank people involved in the dissemination of the CLEARSY Safety Platform, allowing us to deliver talks, courses and hands-on sessions to their colleagues, either teachers, researchers or students, and in particular: Alexander Romanovsky (Univ. Newcastle, UK), Christiano Braga (UFF, Brazil), Emmanuel Chailloux (LIP6, France), Marc Frappier (Univ. Sherbrooke, Canada), Idir Ait Sadoune (CentraleSupelec, France), José Oliveira (Univ. Minho, Portugal), Leopoldo Teixeira (Univ. Pernambuco, Brazil), Marcel Oliveira (UFRN, Brazil), Tiago Massoni (Univ. Campina Grande, Brazil), Valerio Medeiros (IFRN, Brazil), and Yamine Ait-Ameur (Enseeiht, France). \\

\begin{figure}[h]
\centering\includegraphics[scale=0.3]{Pictures/FOREWORD-TALK.jpg}
\caption{A busy hands-on session at the University of Minho in Braga, Portugal}
\end{figure}


Particular thanks are due to the team in charge of its development over the past years: Adrien Somoza, Bruno Lavaud, David Deharbe, Denis Sabatier, Eliott Trotebas, Emine Aktepe, Etienne Prun, Florent Patin, Guillaume Pressouyre, Hector Ruiz Barradas, José Tarsitano, Lilian Burdy, Loïc Claudet, Ludovic Delfau, Manfred Winkler, Mathieu Comptier, Maxime Renaud, Patrick Sauvage, Sébastien Agostini, Sylvain Breux, Thierry Lecomte, Thomas Gonthier, Vivien Galuchot.\\

%
\hspace*{\fill} Aix en Provence, France, July 2019


%---------------------------------------------------------------------
%	INTRODUCTION
%---------------------------------------------------------------------
\chapterimage{back2.jpg} % Chapter heading image
\chapter{Introduction}
\chapterimage{back2.jpg} % Chapter heading image
\chapter{Introduction}

Developing safety critical applications often requires rare human resources to complete successfully, while off-the-shelf block solutions appear difficult to adapt especially during short-term projects. The CLEARSY Safety Platform fulfils a need for a technical solution to overcome the difficulties of developing SIL3/SIL4 system with its technology based on a double-processor and a formal method with proof to ensure safety at the highest level\cite{lecomte2016double}. The formal method, namely the B method\cite{Abrial.1996}, has been heavily used in the railway industry for decades\cite{DBLP:conf/fmics/Lecomte09}\cite{DBLP:conf/fm/Lecomte08}\cite{DBLP:journals/entcs/Benveniste11}. Using its IDE, Atelier B, to program the CLEARSY Safety Platform ensures a higher level of confidence in the generated software.\\

\begin{figure}[h]
\centering\includegraphics[scale=0.5]{Pictures/INTRO-AtelierB.jpg}
\caption{Metros and trains equipped with B SIL4 software}
\end{figure}

The CLEARSY Safety Platform is both a software and a hardware platform aimed at designing and executing safety critical applications. One formal modelling language is used to program the board. Programs are developed using the dedicated IDE or could be the by-product of some translation from a Domain Specific Language to B. The IDE takes care of the verification of the software (type check, proof, compilation) and then ensures its upload to the hardware platform. The program is guaranteed to execute until a misbehaviour is detected, leading to a safe restricted mode where board outputs are deactivated.\\

The CLEARSY Safety Platform eases the development of safety critical applications as:
\begin{itemize}
    \item it covers the whole development cycle,
    \item the safety principles are built-in and are out of reach of the developer, who cannot alter them,
    \item it is based on a formal language (B) and related proof tools,
    \item the mathematical proof replaces unit and integration testing.\\
\end{itemize}

%
The CLEARSY Safety Platform eases the certification of safety critical applications as:
\begin{itemize}
    \item the safety cannot be altered by the developer,
    \item it will come with a certification kit.\\
\end{itemize}

The building blocks of the CLEARSY Safety Platform, already certified in international projects during the years 2017 and 2018 by several certification bodies, have been used to develop a generic version of this technology that could fit a broader range of applications. 

\begin{figure}[h]
\centering\includegraphics[scale=0.25]{Pictures/INTRO-SK0+SK1.jpg}
\caption{Starter kits SK$_0$ (left) and SK$_1$ (right)}
\end{figure}

The first starter kit, SK$_0$, allows to experiment with the whole development chain, including the IDE, using the B-Method and an electronic board hosting the safe execution platform relying on two PIC32 microcontrollers, providing 3 digital inputs and 2 digital outputs.\\

The second starter kit, SK$_1$, is functionally identical to SK$_0$. It provides 20 digital inputs and 8 digital outputs. The core automaton, with its two PIC32 microcontrollers, is hosted on a motherboard while the inputs/outputs are located on a daughterboard.


%=======================================================================
%	PART I: STRATEGY
%=======================================================================

\part{Strategy}

%---------------------------------------------------------------------
%	Chapter Why using B ?
%---------------------------------------------------------------------

\chapterimage{back2.jpg} % Chapter heading image
\chapter{Why using B ?}


%----------------------------------------------------------------------------------------
%	CHAPTER 2
%----------------------------------------------------------------------------------------

\chapter{In-text Elements}

\section{Theorems}\index{Theorems}

This is an example of theorems.

\subsection{Several equations}\index{Theorems!Several Equations}
This is a theorem consisting of several equations.

\begin{theorem}[Name of the theorem]
In $E=\mathbb{R}^n$ all norms are equivalent. It has the properties:
\begin{align}
& \big| ||\mathbf{x}|| - ||\mathbf{y}|| \big|\leq || \mathbf{x}- \mathbf{y}||\\
&  ||\sum_{i=1}^n\mathbf{x}_i||\leq \sum_{i=1}^n||\mathbf{x}_i||\quad\text{where $n$ is a finite integer}
\end{align}
\end{theorem}

\subsection{Single Line}\index{Theorems!Single Line}
This is a theorem consisting of just one line.

\begin{theorem}
A set $\mathcal{D}(G)$ in dense in $L^2(G)$, $|\cdot|_0$. 
\end{theorem}

%------------------------------------------------

\section{Definitions}\index{Definitions}

This is an example of a definition. A definition could be mathematical or it could define a concept.

\begin{definition}[Definition name]
Given a vector space $E$, a norm on $E$ is an application, denoted $||\cdot||$, $E$ in $\mathbb{R}^+=[0,+\infty[$ such that:
\begin{align}
& ||\mathbf{x}||=0\ \Rightarrow\ \mathbf{x}=\mathbf{0}\\
& ||\lambda \mathbf{x}||=|\lambda|\cdot ||\mathbf{x}||\\
& ||\mathbf{x}+\mathbf{y}||\leq ||\mathbf{x}||+||\mathbf{y}||
\end{align}
\end{definition}

%------------------------------------------------

\section{Notations}\index{Notations}

\begin{notation}
Given an open subset $G$ of $\mathbb{R}^n$, the set of functions $\varphi$ are:
\begin{enumerate}
\item Bounded support $G$;
\item Infinitely differentiable;
\end{enumerate}
a vector space is denoted by $\mathcal{D}(G)$. 
\end{notation}

%------------------------------------------------

\section{Remarks}\index{Remarks}

This is an example of a remark.

\begin{remark}
The concepts presented here are now in conventional employment in mathematics. Vector spaces are taken over the field $\mathbb{K}=\mathbb{R}$, however, established properties are easily extended to $\mathbb{K}=\mathbb{C}$.
\end{remark}

%------------------------------------------------

\section{Corollaries}\index{Corollaries}

This is an example of a corollary.

\begin{corollary}[Corollary name]
The concepts presented here are now in conventional employment in mathematics. Vector spaces are taken over the field $\mathbb{K}=\mathbb{R}$, however, established properties are easily extended to $\mathbb{K}=\mathbb{C}$.
\end{corollary}

%------------------------------------------------

\section{Propositions}\index{Propositions}

This is an example of propositions.

\subsection{Several equations}\index{Propositions!Several Equations}

\begin{proposition}[Proposition name]
It has the properties:
\begin{align}
& \big| ||\mathbf{x}|| - ||\mathbf{y}|| \big|\leq || \mathbf{x}- \mathbf{y}||\\
&  ||\sum_{i=1}^n\mathbf{x}_i||\leq \sum_{i=1}^n||\mathbf{x}_i||\quad\text{where $n$ is a finite integer}
\end{align}
\end{proposition}

\subsection{Single Line}\index{Propositions!Single Line}

\begin{proposition} 
Let $f,g\in L^2(G)$; if $\forall \varphi\in\mathcal{D}(G)$, $(f,\varphi)_0=(g,\varphi)_0$ then $f = g$. 
\end{proposition}

%------------------------------------------------

\section{Examples}\index{Examples}

This is an example of examples.

\subsection{Equation and Text}\index{Examples!Equation and Text}

\begin{example}
Let $G=\{x\in\mathbb{R}^2:|x|<3\}$ and denoted by: $x^0=(1,1)$; consider the function:
\begin{equation}
f(x)=\left\{\begin{aligned} & \mathrm{e}^{|x|} & & \text{si $|x-x^0|\leq 1/2$}\\
& 0 & & \text{si $|x-x^0|> 1/2$}\end{aligned}\right.
\end{equation}
The function $f$ has bounded support, we can take $A=\{x\in\mathbb{R}^2:|x-x^0|\leq 1/2+\epsilon\}$ for all $\epsilon\in\intoo{0}{5/2-\sqrt{2}}$.
\end{example}

\subsection{Paragraph of Text}\index{Examples!Paragraph of Text}

\begin{example}[Example name]
\lipsum[2]
\end{example}

%------------------------------------------------

\section{Exercises}\index{Exercises}

This is an example of an exercise.

\begin{exercise}
This is a good place to ask a question to test learning progress or further cement ideas into students' minds.
\end{exercise}

%------------------------------------------------

\section{Problems}\index{Problems}

\begin{problem}
What is the average airspeed velocity of an unladen swallow?
\end{problem}

%------------------------------------------------

\section{Vocabulary}\index{Vocabulary}

Define a word to improve a students' vocabulary.

\begin{vocabulary}[Word]
Definition of word.
\end{vocabulary}

\input{chapterIntroToProof/intro_to_proof.tex}

%----------------------------------------------------------------------------------------
%	PART
%----------------------------------------------------------------------------------------

\part{Development}

%----------------------------------------------------------------------------------------
%	CHAPTER 3
%----------------------------------------------------------------------------------------

\chapterimage{back2.jpg} % Chapter heading image

\chapter{Presenting Information}

\section{Table}\index{Table}

\begin{table}[h]
\centering
\begin{tabular}{l l l}
\toprule
\textbf{Treatments} & \textbf{Response 1} & \textbf{Response 2}\\
\midrule
Treatment 1 & 0.0003262 & 0.562 \\
Treatment 2 & 0.0015681 & 0.910 \\
Treatment 3 & 0.0009271 & 0.296 \\
\bottomrule
\end{tabular}
\caption{Table caption}
\end{table}

\lstset{frameround=fttt}

\begin{lstlisting}[language=B,frame=trBL]
MACHINE
    Access_Consistency
SETS OPTION; PRINTER    
ASSERTIONS
    #(options, pp , oo). 
        (options : PRINTER <-> OPTION 
            & dom(options) = PRINTER 
            & ran(options) = OPTION
            & pp = PRINTER
            & oo = OPTION)

END
\end{lstlisting}

%------------------------------------------------

\section{Figure}\index{Figure}

\begin{figure}[h]
\centering\includegraphics[scale=0.5]{placeholder}
\caption{Figure caption}
\end{figure}

%----------------------------------------------------------------------------------------
%	BIBLIOGRAPHY
%----------------------------------------------------------------------------------------

\chapter*{Bibliography}
\addcontentsline{toc}{chapter}{\textcolor{ocre}{Bibliography}}
\section*{Books}
\addcontentsline{toc}{section}{Books}
\printbibliography[heading=bibempty,type=book]
\section*{Articles}
\addcontentsline{toc}{section}{Articles}
\printbibliography[heading=bibempty,type=article]

%----------------------------------------------------------------------------------------
%	INDEX
%----------------------------------------------------------------------------------------

\cleardoublepage
\phantomsection
\setlength{\columnsep}{0.75cm}
\addcontentsline{toc}{chapter}{\textcolor{ocre}{Index}}
\printindex

%----------------------------------------------------------------------------------------


\lstset{frameround=fttt,keywordstyle=\color{ocre}\bfseries}
\lstinputlisting[language=B, frame=trbl, lastline=20,caption={Useless code},label=useless, rulecolor=\color{ocre}]{chapterIntroToProof/b_util.mch}

This is an example of theorems.
\lstset{keywordstyle=\color{ocre}\bfseries}
\lstinline[language=B]!BEGIN ii : INTEGER END!

See listing \ref{useless}
\lstset{keywordstyle=\color{black}, backgroundcolor=\color{anti-flashwhite}}
\lstinputlisting[language=C, frame=single, framerule=0pt, firstline=3, lastline=23, basicstyle=\small]{chapterIntroToProof/L0_i.c}

To demonstrate the proof use the following demonstration: 

\lstset{keywordstyle=\color{black}\bfseries}
\lstinline[language=Proof]!dd(0) & dc(xx, BOOL) & mp & mp!

\subsection{Several equations}\index{Theorems!Several Equations}
This is a theorem consisting of several equations.

\begin{theorem}[Name of the theorem]
In $E=\mathbb{R}^n$ all norms are equivalent. It has the properties:
\begin{align}
& \big| ||\mathbf{x}|| - ||\mathbf{y}|| \big|\leq || \mathbf{x}- \mathbf{y}||\\
&  ||\sum_{i=1}^n\mathbf{x}_i||\leq \sum_{i=1}^n||\mathbf{x}_i||\quad\text{where $n$ is a finite integer}
\end{align}
\end{theorem}

\subsection{Single Line}\index{Theorems!Single Line}
This is a theorem consisting of just one line.



\begin{theorem}
A set $\mathcal{D}(G)$ in dense in $L^2(G)$, $|\cdot|_0$. 
\end{theorem}

%------------------------------------------------

\section{Proof 1}\index{Proof 1}

This is an example of a definition. A definition could be mathematical or it could define a concept.

\begin{definition}[Definition name]
Given a vector space $E$, a norm on $E$ is an application, denoted $||\cdot||$, $E$ in $\mathbb{R}^+=[0,+\infty[$ such that:
\begin{align}
& ||\mathbf{x}||=0\ \Rightarrow\ \mathbf{x}=\mathbf{0}\\
& ||\lambda \mathbf{x}||=|\lambda|\cdot ||\mathbf{x}||\\
& ||\mathbf{x}+\mathbf{y}||\leq ||\mathbf{x}||+||\mathbf{y}||
\end{align}
\end{definition}

%------------------------------------------------

\section{Proof 2}\index{Proof 2}

\begin{notation}
Given an open subset $G$ of $\mathbb{R}^n$, the set of functions $\varphi$ are:
\begin{enumerate}
\item Bounded support $G$;
\item Infinitely differentiable;
\end{enumerate}
a vector space is denoted by $\mathcal{D}(G)$. 
\end{notation}

%------------------------------------------------

\section{Proof 3}\index{Proof 3}

This is an example of a remark.

\begin{remark}
The concepts presented here are now in conventional employment in mathematics. Vector spaces are taken over the field $\mathbb{K}=\mathbb{R}$, however, established properties are easily extended to $\mathbb{K}=\mathbb{C}$.
\end{remark}

%------------------------------------------------

\section{Proof 4}\index{Proof 4}

This is an example of a corollary.

\begin{corollary}[Corollary name]
The concepts presented here are now in conventional employment in mathematics. Vector spaces are taken over the field $\mathbb{K}=\mathbb{R}$, however, established properties are easily extended to $\mathbb{K}=\mathbb{C}$.
\end{corollary}

%------------------------------------------------

\section{Proof 5}\index{Proof 5}

This is an example of propositions.

\subsection{Several equations}\index{Propositions!Several Equations}

\begin{proposition}[Proposition name]
It has the properties:
\begin{align}
& \big| ||\mathbf{x}|| - ||\mathbf{y}|| \big|\leq || \mathbf{x}- \mathbf{y}||\\
&  ||\sum_{i=1}^n\mathbf{x}_i||\leq \sum_{i=1}^n||\mathbf{x}_i||\quad\text{where $n$ is a finite integer}
\end{align}
\end{proposition}

\subsection{Single Line}\index{Propositions!Single Line}

\begin{proposition} 
Let $f,g\in L^2(G)$; if $\forall \varphi\in\mathcal{D}(G)$, $(f,\varphi)_0=(g,\varphi)_0$ then $f = g$. 
\end{proposition}

%------------------------------------------------

\section{Proof 6}\index{Proof 6}

This is an example of examples.

\subsection{Equation and Text}\index{Examples!Equation and Text}

\begin{example}
Let $G=\{x\in\mathbb{R}^2:|x|<3\}$ and denoted by: $x^0=(1,1)$; consider the function:
\begin{equation}
f(x)=\left\{\begin{aligned} & \mathrm{e}^{|x|} & & \text{si $|x-x^0|\leq 1/2$}\\
& 0 & & \text{si $|x-x^0|> 1/2$}\end{aligned}\right.
\end{equation}
The function $f$ has bounded support, we can take $A=\{x\in\mathbb{R}^2:|x-x^0|\leq 1/2+\epsilon\}$ for all $\epsilon\in\intoo{0}{5/2-\sqrt{2}}$.
\end{example}

\subsection{Paragraph of Text}\index{Examples!Paragraph of Text}

\begin{example}[Example name]
\lipsum[2]
\end{example}

%------------------------------------------------

\section{Proof 7}\index{Proof 7}

This is an example of an exercise.

\begin{exercise}
This is a good place to ask a question to test learning progress or further cement ideas into students' minds.
\end{exercise}

%------------------------------------------------

\section{Proof 8}\index{Proof 8}

\begin{problem}
What is the average airspeed velocity of an unladen swallow?
\end{problem}

%------------------------------------------------

\section{Proof 9}\index{Proof 9}

Define a word to improve a students' vocabulary.

\begin{vocabulary}[Word]
Definition of word.
\end{vocabulary}


\end{document}